\subsection{Definición y propiedades fundamentales}

Existen caracterizaciones (véase Números, Grupos y Anillos, página 246) que aconsejan definir el siguiente concepto en el ambiente de los dominios de integridad. 

\begin{definition}[Dominio de ideales principales]
Dado un dominio de integridad $A$. 

$A$ es un dominio de ideales principales si todo ideal de $A$ es principal, esto es, $\forall I < A. \exists x \in A. I = \langle x \rangle$. 
\end{definition}

\begin{theorem}[Teorema de Bézout]
Dado un dominio de ideales principales $A$. 

1. $\forall a,b \in A \exists d=(a,b)$.\\
2. $\exists u,v \in A. d = au+bv$. 

A cualquier pareja $u,v$ que verifique la segunda ecuación se les llama coeficientes de Bézout. 
\end{theorem}
\begin{proof}
Dados $a,b \in A$, consideremos $\langle a,b \rangle = \{ax+by:x,y \in A\}$ este es el ideal generado por $a$ y $b$, esto es, el menor ideal que los contiene. En efecto, como $a,b \in \langle a,b \rangle$ sabemos que $\langle a,b \rangle$ no es vacío. También es claro que $\langle a,b \rangle$ es un ideal ya que $ax+by + ax' +by' = a(x+x')+b(y+y') \in I$ y $c(ax+by) = a(xc)+b(yc) \in I$. 

Utilizamos que $A$ es un dominio de ideales principales y obtenemos que debe existir $d \in A.\langle a,b \rangle = \langle d \rangle$ y por tanto $d = au+bv$ para convenientes $u,v \in A$. Por tanto, hemos deducido la segunda propiedad.

Por lo anterior, $d|a \land d|b$ y si $c \in A$ verifica que $c|a \land c|b$ entonces $c|au+bv$ de donde $c|d$. Esto nos dice que $d = (a,b)$
\end{proof}

\begin{example}[Un ejemplo de anillo de enteros cuadráticos que no es DIP]
Por los ejemplos anteriores como en $\mathbb{Z}[\sqrt{-5}]$ no existe el máximo común divisor de cualesquiera dos elementos tampoco puede ser un dominio de ideales principales. 
\end{example}

\begin{proposition}[Todo DE es un DIP]
Todo dominio euclídeo es un dominio de ideales principales donde cada ideal está generado por el elemento con valor mínimo de la función euclídea.  
\end{proposition}
\begin{proof}
Si $\phi$ es la función euclídea asociada al dominio y consideremos un ideal cualquiera $I \neq \{0\}$. El conjunto $\phi(I \setminus \{0\})$ es un subconjunto no vacío de números naturales y por tanto tiene mínimo. Sea $b$ este mínimo. Demostraremos que $I = \langle b \rangle$. 

$\subseteq)$ Dado que $b \in I \implies \langle b \rangle \subseteq I$. \\
$\supseteq)$ Dado $a \in I \setminus \{0\}$ tenemos que $\phi(a) \ge \phi(b)$. Por estar en un dominio euclídeo, $a = bq + r$ con $r = 0 \lor \phi(r) < \phi(b)$. Si $r = 0$ hemos acabado ya que entonces $a = bq \in \langle b \rangle$ y por tanto $I \subseteq \langle b \rangle$. Si $r \neq 0$ entonces necesariamente $\phi(r) < \phi(b)$. Pero esto contradice que $b$ sea el elemento de valor mínimo del conjunto anterior, ya que $r = a - bq \in I$ y $\phi(r) < \phi(b)$. 
\end{proof}

\begin{example}[Un DIP que no es DE]
$\mathbb{Z}[\frac{1+\sqrt{-19}}{2}]$ es un DIP pero no es DE.
\end{example}

\subsection{Retículo de divisibilidad y de ideales.}

\begin{corollary}[Existencia del mínimo común múltiplo en DIP]
En cualquier dominio de ideales principales (en particular, en los dominios euclídeos), existe el mínimo común múltiplo de cualquiera dos elementos.
\end{corollary}
\begin{proof}
En efecto, en un DIP por el teorema de Bézout existe el máximo común divisor de cualquier par de elementos y como un DIP es un dominio de integridad se tiene por la proposición anterior que existe el mínimo común múltiplo de cualquier par de elementos. 

Cualquier dominio euclídeo es un dominio de ideales principales. 
\end{proof}

\begin{corollary}[Retículo de divisibilidad]
Sea $D$ un dominio de ideales principales y sea $(,),[,]$ el máximo común divisor y el mínimo común múltiplo respectivamente. Entonces $(D,(,),[,])$ es un retículo.
\end{corollary}
\begin{proof}
Elijamos $[,]$ como ínfimo y $(,)$ como supremo. 

Por lo anterior, estas operaciones son internas en $D$. 

Claramente, se verifican las propiedades conmutativa, asociativa y la propiedad de idempotencia. Faltaría demostrar la propiedad de absorción que diría $[x,(x,y)] = x$ y $(x,[x,y]) = x$. 

Por otro lado, el máximo del retículo es el $1$ del dominio y el mínimo del retículo es el $0$ del dominio.

En consecuencia, $(a,1) = 1,(a,0) = a$. También el $0$ del dominio es el mínimo del retículo, y en particular, $[a,0] = 0,[x,1] = x$.
\end{proof}

\begin{tikzcd}
1    \\
x \arrow{u}{mcd} \arrow{d}{mcm}  \\   
0 
\end{tikzcd}


Pregunta: ¿este retículo es distributivo, es complementado? ¿Es el 1 el máximo y el 0 el mínimo o hay maximales (las unidades) y minimales? Aquí solo hace falta que sea distributivo y complementado para ser un álgebra de Boole. Ojo: quien es un retículo es $D/\sim$ donde $\sim$ es la relación de ser asociados. 

\begin{corollary}[Retículo de ideales de un DIP]
Sea $A$ un dominio de ideales principales consideremos el retículo de ideales ordenado por la inclusión las operaciones supremo e ínfimo y producto están definidas para este en términos del máximo común divisor y del mínimo común múltiplo. Más precisamente, para todo $a,b \in A$:

\begin{enumerate}
\item $\langle a \rangle + \langle b \rangle = \langle (a,b) \rangle$
\item $\langle a \rangle \cap \langle b \rangle = \langle [a,b] \rangle$
\item $\langle a \rangle \cdot \langle b \rangle = \langle ab \rangle$
\end{enumerate}
\end{corollary}
\begin{proof}
\begin{enumerate}
\item Es consecuencia del teorema de Bézout. 
\item Se deriva desde la definición. 
\item Se deriva desde la definición. 
\end{enumerate}
\end{proof}

\begin{proposition}[Ideales principales maximales son los generados por irreducibles]
Sea $A$ un dominio de ideales principales y $a \in A \setminus \{0\}$:

$\langle a \rangle$ es maximal $\iff a$ es irreducible.
\end{proposition}
\begin{proof}
$\Rightarrow)$ Si $\langle a \rangle$ es maximal y $a = bc$ entonces como todo ideal maximal es primo y $bc \in \langle a \rangle$ se tendría que $b \in \langle a \rangle \lor c \in \langle a \rangle$. Supongamos que $b \in \langle a \rangle$, entonces $b = ad = bcd$ y como $b \neq 0$ se simplifica a $1 = cd$ de modo que $c \in U(A)$ y la factorización es impropia de modo que $a$ es irreducible. 

$\Leftarrow)$ Supongamos que $a$ es irreducible y veamos que $\langle a \rangle$ es maximal. En efecto, como $a$ es irreducible, no es una unidad y por tanto, $\langle a \rangle \neq A$. Por otro lado, si $\langle a \rangle \subseteq \langle b \rangle \neq A$  entonces $b|a$ y $b \notin U(A), b \neq 0$, luego tenemos que $a = bc$ donde $c \neq 0$ y $c$ debe ser una unidad ya que no existen factorizaciones propias de $a$. Por tanto, $\langle a \rangle = \langle b \rangle$.
\end{proof}


\subsection{Solución de ecuaciones diofánticas y algoritmo de Euclides}

\begin{theorem}[Resolución de ecuaciones diofánticas lineales en un DIP]
Sea $A$ un dominio de ideales principales y $a,b,c \in A$ con $a,b \neq 0$. 

1. La ecuación diofántica lineal $ax+by = c$ tiene solución $\iff d|c$ con $d = (a,b)$. \\
2. Si $d|c$ y $(x_0,y_0)$ es una solución particular entonces la solución general da para cada $k \in A$ la solución $(x_0+k\frac{b}{d},y_0-k\frac{a}{d})$. 
\end{theorem}
\begin{proof}
\begin{enumerate}
\item La ecuación tiene solución $\iff c \in \langle a,b \rangle = \langle d \rangle \iff d|c$ con $d = (a,b)$. Donde hemos utilizado el corolario al teorema de Bézout.
\item Supongamos que $x_0,y_0$ es una solución particular. Claramente, para cada $k \in A$ la pareja $(x_0+k\frac{b}{d},y_0-k\frac{a}{d})$ es una solución particular ya que $a(x_0+k\frac{b}{d})+b(y_0-k\frac{a}{d}) = ax_0+by_0 = c$. 

Veamos que no hay más soluciones. Si $(x,y)$ es otra solución. Entonces restando las ecuaciones para $(x,y)$ y $(x_0,y_0)$ obtenemos $a(x-x_0)+b(y-y_0) = 0$, esto es, $a(x-x_0) = -b(y-y_0)$. Esto nos dice que $\frac{b}{d}$ divide a $\frac{a}{d}(x-x_0)$ y por el lema de Euclides como $(\frac{b}{d},\frac{a}{d}) = 1$ se verificará qque $\frac{b}{d}|(x-x_0)$. Por tanto, existe $k \in A$ tal que $x-x_0 = k\frac{b}{d}$ o equivalentemente $x = x_0 + k\frac{b}{d}$ como queríamos. Análogamente, existe un $h \in A$ tal que $y = y_0 - h \frac{a}{d}$. 

Pero resulta que $k = h$. Esto se puede ver sustituyendo en la ecuación $a(x-x_0)+b(y-y_0) = 0$ llegando a que $ab(k-h) = 0$. Dado que $a,b \neq 0$ y que $A$ es un dominio de integridad, $x = h$. 
\end{enumerate}
\end{proof}

El teorema anterior no da un método para calcular la solución particular $(x_0,y_0)$. 

Sabemos que $c = d \cdot c'$ por la condición de existencia de solución. Por otro lado, como $d = au+bv$ por la identidad de Bézout, tendríamos que $c = a(uc')+b(vc')$. Entonces elegimos $x_0 = uc' \land y_0 = vc'$. 

Es claro que en el anterior algoritmo necesitamos conocer $d,u,v$. El siguiente algoritmo, nos da los coeficientes de Bézout, el máximo común divisor y el mínimo común múltiplo. 

\begin{theorem}[Algoritmo extendido de Euclides]
Dado $A$ un dominio euclídeo y $a,b \in A$. 

\begin{enumerate}
\item Si $b = 0$ hemos acabado ya que $(a,0) = a$, $u = 1,v= 0$. 
\item Si $a = 0$ hemos acabado ya que $(0,b) = b$, $u = 0,v= 1$. 
\item Supongamos $a,b \neq 0$ y $\phi(a) \ge \phi(b)$. Si la desigualdad anterior no se da siempre podemos intercambiar $a$ y $b$ ya que $(a,b) = (b,a)$. 
\begin{itemize}
\item Dividir $a$ entre $b$ para obtener $a = bq_1+r_1$ con $r_1 = 0 \lor \phi(r_1) < \phi(b)$. 
\item Si $r_1 = 0 \implies (a,b) = b$.
\item Si $r_1 \neq 0 \implies (a,b) = (a-qb,b) = (b,r_1)$ . 
\end{itemize}
\item Análogamente se continúa obteniendo la lista de ecuaciones $$b = r_1q_2+r_2$$ $$r_1 = r_2q_3+r_3$$ $$\ldots$$ $$r_n = r_{n+1}q_{n+2}+r_{n+2}$$ Donde $r_{n+1} = 0$ y $r_n = (a,b) = (0,r_n)$. Se puede asegurar que llegaremos a este punto mediante el método de descenso infinito aplicado a la función euclídea de los restos. 

\item Para obtener los coeficientes de Bézout en cada paso se realiza el siguiente cálculo. 

Si $\alpha = au+bv$ y $\alpha' = au'+bv'$ entonces: $$\alpha'' = \alpha - \alpha'q = a(u-qu') + b(v-qv') = au'' + bv''$$ Finalmente, los coeficientes que se obtienen para $r_n$ son los coeficientes de Bézout para $a,b$ ya que $r_n = (a,b)$. De forma indirecta, $[a,b] = u_{n+1}a = v_{n+1}b$. 
\end{enumerate}

El procedimiento puede ser resumido en la siguiente tabla:

\begin{center}
  \begin{tabular}{ | l | c | r |}
    \hline
    $\alpha$ & u & v \\ \hline
    a & 1 & 0 \\ \hline
    b & 0 & 1 \\ \hline
    $r_1$ & $1$ & $-q_1$ \\ \hline
    $r_2$ & $-q_2$ & $1+q_1q_2$ \\ \hline
    $r_3$ & $1+q_1q_3$ & $-q_1-q_3-q_1q_2q_3$ \\ \hline
    $\cdots$ & $\cdots$ & $\cdots$ \\ \hline 
    $r_n$ & $u_n$ & $v_n$ \\ \hline
    $r_{n+1} = 0$ & $u_{n+1}$ & $v_{n+1}$ \\ 
    \hline
  \end{tabular}
\end{center}

Es importante darse cuenta como se obtienen los sucesivos coeficientes de Bézout en esta tabla. Se pasa de una fila a otra tomando las dos anteriores y restando a la última de ellas la anterior multiplicada por el cociente obtenido al dividir los correspondientes términos de la columna de la izquierda. 

Hacemos varias observaciones al método general. 

Primeramente es necesario observar que en cada paso, $r_i = a u_i + b v_i$ por construcción. 

La igualdad para el mínimo común múltiplo se deduce observando que la sucesión que hace las diferencias en cruz de productos de la primera fila y la segunda: $$w_i = u_i r_{i+1} - u_{i+1} r_i$$ es una sucesión constante que alterna el signo. En efecto: $$w_i = u_i r_{i+1} - s_{i+1} u_i = u_i (r_{i-1} - q_ir_i) - (u_{i-1} - q_i u_i) r_i = - (u_{i-1} r_i - u_i r_{i-1})$$ Como $w_0 = u_0 r_1 - u_1 r_0 = b$, se tendrá que $w_i = (-1)^i w_0$ y finalmente, para como $r_{n+1} = 0$, se tendría: $$w_{n+1} = (-1)^{n+1} b = - u_{n+1} mcd(a,b)$$ es decir: $$ab = (-1)^n u_{n+1} a (a,b)$$ Como estamos en un dominio de integridad, necesariamente, la ecuación tiene una única solución que es $[a,b] = u_{n+1}a$ donde recordamos que el mínimo común múltiplo es único salvo asociados. 
\end{theorem}

\begin{example}[Ejemplos de aplicación del algoritmo]
\begin{enumerate}
\item Aplicar el algoritmo de Euclides para $a = 30,b = 12$:

\begin{center}
  \begin{tabular}{ | l | c | r |}
    \hline
    $\alpha$ & u & v \\ \hline
    30 & 1 & 0 \\ \hline
    12 & 0 & 1 \\ \hline
    $6$ & $1$ & $-2$ \\ \hline
    $0$ & $-2$ & $5$ \\ \hline
  \end{tabular}
\end{center}

Como consecuencia, $(30,12) = 6$ y $[30,12] = -2 \cdot 30 = -60 \sim 60 = 5 \cdot 12$ y los coeficientes de Bézout son $1,-2$. 

\item Aplicar el algoritmo de Euclides para $a = 3+2i,b = 2-3i$:

Recordando cómo se divide en $\mathbb{Z}[i]$ tendríamos que:

\begin{center}
  \begin{tabular}{ | l | c | r |}
    \hline
    $\alpha$ & u & v \\ \hline
    $3+2i$ & 1 & 0 \\ \hline
    $2-3i$ & 0 & 1 \\ \hline
    $0$ & $1$ & $i$ \\ \hline
  \end{tabular}
\end{center}

Como consecuencia, $(3+2i,2-3i) = 2-3i$ y $[3+2i,2-3i] = 3+2i \cdot 1 = 3+2i = i \cdot 2-3i$ y los coeficientes de Bézout son $0,1$. 
\end{enumerate}
\end{example}

